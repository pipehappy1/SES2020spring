\documentclass{article}
\usepackage{graphicx}
\usepackage{amsmath}
\usepackage{hyperref}
\usepackage{url}
\usepackage{caption}
\usepackage{subcaption}
\usepackage{mathtools}


\begin{document}
\title{Autorouting with deep learning}
\maketitle

{\textbf Problem statement:}
On a $(w=5, l=5)$ board, there are $n=30$ components.
The width $w$ and length $l$ of these components follows
a Gaussian distribution of $N(\mu=0.5, \sigma^2=0.5)$ indepedently.
All pads/holes are even spaced and aligned at the two sides of these components, as DIP.
The number of pads for one component
follows a Poisson distribution with $P(\lambda = 6)$ per unit length of the component perimeter (You can remove zero and single pad samples).
The netlist is formed by a partition among all the pads.
With fixed number of partitions $k$, probability for each partition $\pi = (\pi_1, \dots, \pi_K)$ is sampled from the Dirichlet distribution with parameter $(\theta_1, \dots, \theta_k)$.
Then each pad takes values in $(1, \dots, k)$ withe probability $\pi$.
This is the Dirichlet-multinomial random partition generation.

{\textbf Layout the route:}
Your task is to build a model to route the netlist by restrictions, such as no route sould be crossed.
You can have multiple layer PCB board and vias.



 \end{document}


\begin{description}
\item [item 1] say 1
\item [item 2] say 2
\item [item 3] say 3
\end{description}
  

\begin{figure*}
  \centering
  \begin{subfigure}[b]{0.5\textwidth}
    \centering
    \includegraphics[width=0.9\textwidth]{fig/theme_relu_small.jpg}
    \caption{ReLU in a convolutional neural network}
    \label{fig:relunet}
  \end{subfigure}%
  ~
  \begin{subfigure}[b]{0.5\textwidth}
    \centering
    \includegraphics[width=0.9\textwidth]{fig/theme_sin_small.jpg}
    \caption{Sin in a convolutional neural network}
    \label{fig:sinnet}
  \end{subfigure}
% figure caption is below the figure
  \caption{One layer of neural network is shown in the figure. The input is feed through
    convolutional filter and followed by a nonlinear activation unit.
    The convolutional neural neural network is often used with ReLU
    and we suggests to use periodic function such as sin function shown in the right figure
  as the nonlinear unit.}
\label{fig:theme}       % Give a unique label
